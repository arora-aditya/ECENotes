\documentclass[11pt]{article}
\usepackage[utf8]{inputenc}
\usepackage{amsmath,amsthm,amsfonts,amssymb,amscd}
\usepackage{multirow,booktabs}
\usepackage[table]{xcolor}
\usepackage{fullpage}
\usepackage{lastpage}
\usepackage{enumitem}
\usepackage{fancyhdr}
\usepackage{mathrsfs}
\usepackage{wrapfig}
\usepackage{setspace}
\usepackage{calc}
\usepackage{multicol}
\usepackage{cancel}
\usepackage[retainorgcmds]{IEEEtrantools}
\usepackage[margin=3cm]{geometry}
\usepackage{amsmath}
\newlength{\tabcont}
\setlength{\parindent}{0.0in}
\setlength{\parskip}{0.05in}
\usepackage{empheq}
\usepackage{framed}
\usepackage[most]{tcolorbox}
\usepackage{xcolor}
\colorlet{shadecolor}{orange!15}
\parindent 0in
\parskip 12pt
\geometry{margin=1in, headsep=0.25in}
\theoremstyle{definition}
\newtheorem{defn}{Definition}
\newtheorem{reg}{Rule}
\newtheorem{exer}{Exercise}
\newtheorem{note}{Note}
\usepackage{listings}
\usepackage{xcolor}
\usepackage{graphicx}
\setlist[itemize]{noitemsep, topsep=0pt}
\setlength{\OuterFrameSep}{0pt}
\graphicspath{ {./images/} }
\newcommand{\N}{\mathbb{N}}
\newcommand{\Z}{\mathbb{Z}}
\newcommand{\C}{\mathbb{C}}
\newcommand{\R}{\mathbb{R}}
\newcommand*{\Co}[2]{{}^{#1}C_{#2}}%
\lstset { %
    language=C++,
    backgroundcolor=\color{black!5}, % set backgroundcolor
    basicstyle=\footnotesize,% basic font setting
}
\NewDocumentCommand{\codeword}{v}{%
\texttt{\textbf{#1}}%
}
\begin{document}
\title{Review}
\thispagestyle{empty}

\begin{center}
{\vspace{5mm} \LARGE ECE 205 - Fall 2018 \\ \vspace{5mm}Aditya Arora\\ \vspace{5mm} 6th September 2018}\\
{\vspace{5mm} \LARGE \bf Week 1 - Review \& Introduction}

\end{center}

\section{Complex Numbers}
Mathematics: $"i" = \sqrt{-1}$\\
Engineering: $"j" = \sqrt{-1}$
\subsection{Euler Formula}
$$e^{j\theta} = cos\theta + jsin\theta$$
We will have expression of the form $e^mt$ where m satisfies $am^2 + bm + c = 0$
\subsubsection{Real Roots}
If $am^2 + bm + c = 0$ has real roots $r_1, r_2$ then that leads to solutions $e^{r_1t}, e^{r_2t}$\\
We often take linear combinations of those solutions: $a_1e^{r_1t} + a_2e^{r_2t}$

\subsubsection{Complex Roots}
If $am^2 + bm + c = 0$ has complex roots $\alpha + j\beta, \alpha - j\beta, \;(\alpha,\: \beta\: \epsilon\: \R)$ then that leads to functions $e^{(\alpha + j\beta)t}, e^{(\alpha - j\beta)t}$\\
We also take linear combinations of those solutions: $$\frac{1}{2}[e^{(\alpha + j\beta)t} + e^{(\alpha - j\beta)t}] = e^{\alpha t}cos\beta t$$ because the complex component cancels out from Euler's formula. \\
Similarly taking a complex linear combination, $$\frac{1}{2j}[e^{(\alpha + j\beta)t} - e^{(\alpha - j\beta)t}] = e^{\alpha t}sin\beta t$$
Taking a real linear combination, we will have a solution of the form
$$Ae^{\alpha t}cos\beta t + Be^{\alpha t}sin\beta t \; (A,\:B\: \epsilon\: \R)$$

\section{Linear Algebra}
Everything is set up to manage and use Vector Spaces as $\R^n, M_{m+n} \; P_3$
\subsection{Linear Equation}
$$A\vec{x} = \vec{b}$$
$A\:  \epsilon\:  M_{m+n}\: [known],\: b\: \epsilon\: \R^n\:  [unknown],\: b\: \epsilon\:  \R^m\:  [known]$
\subsubsection{Special Cases}
Homogeneous Equation:$A\vec{x} = \vec{0}_{\R^m}$\\
$\vec{0}_{\R^n}$ is a solution\\
A solution set \textbf{$S$} is a vector subspace of $\R^n$\\
If $\vec{x}_1, \vec{x}_2 \epsilon S$ then $\vec{x}_1 + \vec{x}_2 \epsilon S$ and if $\vec{x}_1 \epsilon S $ then $ c\vec{x}_1 \epsilon S$
\subsubsection{Some other points}
Consider In-homogeneous equation $$A\vec{x} = \vec{b}$$
$\vec{b} \neq \vec{0}$ with  solution set $\tilde{S}$
\begin{enumerate}
    \item Lemma 1: Let $\vec{y_1}\; \epsilon\; \tilde{S}$ and $\vec{y_2}\; \epsilon\; \tilde{S}$\\
    Then $\vec{y_1}\; -\; \vec{y_2}\; \epsilon\; S$\\
    Moreover $\vec{y_1}\; -\; \vec{y_2}\; =\; \vec{x_1}$ where $\vec{x_1}\; \epsilon\; S$ then $\vec{y_1}\; =\; \vec{x_1}\; +\; \vec{y_2}$
    \item Lemma 2: We can obtain $\tilde{S}$ by finding one solution $\vec{y_2}\; \epsilon\; \tilde{S}$ [a particular solution] and then add it to each and every solution in set $S$ one by one\\
    If there are no solutions i.e. you cannot obtain $\vec{y_2}$, then $\tilde{S}$ is empty
\end{enumerate}
\newpage

\end{document}
