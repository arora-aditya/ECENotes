\documentclass[11pt]{article}
\usepackage[utf8]{inputenc}
\usepackage{amsmath,amsthm,amsfonts,amssymb,amscd}
\usepackage{multirow,booktabs}
\usepackage[table]{xcolor}
\usepackage{fullpage}
\usepackage{lastpage}
\usepackage{enumitem}
\usepackage{fancyhdr}
\usepackage{mathrsfs}
\usepackage{wrapfig}
\usepackage{setspace}
\usepackage{calc}
\usepackage{multicol}
\usepackage{cancel}
\usepackage[retainorgcmds]{IEEEtrantools}
\usepackage[margin=3cm]{geometry}
\usepackage{amsmath}
\newlength{\tabcont}
\setlength{\parindent}{0.0in}
\setlength{\parskip}{0.05in}
\usepackage{empheq}
\usepackage{framed}
\usepackage[most]{tcolorbox}
\usepackage{xcolor}
\colorlet{shadecolor}{orange!15}
\parindent 0in
\parskip 12pt
\geometry{margin=1in, headsep=0.25in}
\theoremstyle{definition}
\newtheorem{defn}{Definition}
\newtheorem{reg}{Rule}
\newtheorem{exer}{Exercise}
\newtheorem{note}{Note}
\usepackage{listings}
\usepackage{xcolor}
\usepackage{graphicx}
\setlist[itemize]{noitemsep, topsep=0pt}
\setlength{\OuterFrameSep}{0pt}
\graphicspath{ {./images/} }
\newcommand{\N}{\mathbb{N}}
\newcommand{\Z}{\mathbb{Z}}
\newcommand{\C}{\mathbb{C}}
\newcommand{\R}{\mathbb{R}}
\newcommand*{\Co}[2]{{}^{#1}C_{#2}}%
\lstset { %
    language=C++,
    backgroundcolor=\color{black!5}, % set backgroundcolor
    basicstyle=\footnotesize,% basic font setting
}
\NewDocumentCommand{\codeword}{v}{%
\texttt{\textbf{#1}}%
}
\begin{document}
\title{Review}
\thispagestyle{empty}

\begin{center}
{\vspace{5mm} \LARGE ECE 205 - Fall 2018 \\ \vspace{5mm}Aditya Arora\\ \vspace{5mm} 6th September 2018}\\
{\vspace{5mm} \LARGE \bf Week 1 - Review \& Introduction}

\end{center}

\section{Complex Numbers}
Mathematics: $"i" = \sqrt{-1}$\\
Engineering: $"j" = \sqrt{-1}$
\subsection{Euler Formula}
$$e^{j\theta} = cos\theta + jsin\theta$$
We will have expression of the form $e^mt$ where m satisfies $am^2 + bm + c = 0$
\subsubsection{Real Roots}
If $am^2 + bm + c = 0$ has real roots $r_1, r_2$ then that leads to solutions $e^{r_1t}, e^{r_2t}$\\
We often take linear combinations of those solutions: $a_1e^{r_1t} + a_2e^{r_2t}$

\subsubsection{Complex Roots}
If $am^2 + bm + c = 0$ has complex roots $\alpha + j\beta, \alpha - j\beta, \;(\alpha,\: \beta\: \epsilon\: \R)$ then that leads to functions $e^{(\alpha + j\beta)t}, e^{(\alpha - j\beta)t}$\\
We also take linear combinations of those solutions: $$\frac{1}{2}[e^{(\alpha + j\beta)t} + e^{(\alpha - j\beta)t}] = e^{\alpha t}cos\beta t$$ because the complex component cancels out from Euler's formula. \\
Similarly taking a complex linear combination, $$\frac{1}{2j}[e^{(\alpha + j\beta)t} - e^{(\alpha - j\beta)t}] = e^{\alpha t}sin\beta t$$
Taking a real linear combination, we will have a solution of the form
$$Ae^{\alpha t}cos\beta t + Be^{\alpha t}sin\beta t \; (A,\:B\: \epsilon\: \R)$$

\section{Linear Algebra}
Everything is set up to manage and use Vector Spaces as $\R^n, M_{m+n} \; P_3$
\subsection{Linear Equation}
$$A\vec{x} = \vec{b}$$
$A\:  \epsilon\:  M_{m+n}\: [known],\: b\: \epsilon\: \R^n\:  [unknown],\: b\: \epsilon\:  \R^m\:  [known]$
\subsubsection{Special Cases}
Homogeneous Equation:$A\vec{x} = \vec{0}_{\R^m}$\\
$\vec{0}_{\R^n}$ is a solution\\
A solution set \textbf{$S$} is a vector subspace of $\R^n$\\
If $\vec{x}_1, \vec{x}_2 \epsilon S$ then $\vec{x}_1 + \vec{x}_2 \epsilon S$ and if $\vec{x}_1 \epsilon S $ then $ c\vec{x}_1 \epsilon S$
\subsubsection{Some other points}
Consider In-homogeneous equation $$A\vec{x} = \vec{b}$$
$\vec{b} \neq \vec{0}$ with  solution set $\tilde{S}$
\begin{enumerate}
    \item Lemma 1: Let $\vec{y_1}\; \epsilon\; \tilde{S}$ and $\vec{y_2}\; \epsilon\; \tilde{S}$\\
    Then $\vec{y_1}\; -\; \vec{y_2}\; \epsilon\; S$\\
    Moreover $\vec{y_1}\; -\; \vec{y_2}\; =\; \vec{x_1}$ where $\vec{x_1}\; \epsilon\; S$ then $\vec{y_1}\; =\; \vec{x_1}\; +\; \vec{y_2}$
    \item Lemma 2: We can obtain $\tilde{S}$ by finding one solution $\vec{y_2}\; \epsilon\; \tilde{S}$ [a particular solution] and then add it to each and every solution in set $S$ one by one\\
    If there are no solutions i.e. you cannot obtain $\vec{y_2}$, then $\tilde{S}$ is empty
\end{enumerate}
\newpage
\begin{center}
    {\LARGE 7th September 2018 \\ Lecture 1}
\end{center}
\setcounter{section}{0}
\section{Differential Equations}
\subsection{Introduction}
In calculus we deal with functions sometimes the functions are provided explicitly, however often we only have information about derivatives(s) of the function. \textit{Eg:} Newtown's law in 1 dimension, with constant mass: $$F = \frac{d\vec{p}}{dt}$$
Using momentum as $\vec{p} = m \vec{v} = m\frac{dx}{dt}$
$$F = m\frac{d^2x}{dt^2}$$
\textit{Eg:} one dimensional Motion under gravity, $g = \frac{d^2x}{dt^2} = 9.81ms^{-2}$
$$F = m\vec{a} = m\vec{g}$$

\subsubsection{Definitions and Vocabulary}
There are many types and we need some vocabulary to distinguish between them\\
\textbf{Def \#1:} A differential equation [DE] is an equation which involves the derivative(s) of some unknown function(s).\\
\textbf{Def \#2:} When we have a DE we will have one (or more) function(s) which we are trying to determine. This function(s) is(are) called the dependent variable(s). The other variables are called independent variables.\\
\textit{Eg:} $\frac{d^2x}{dt^2} = g$, $g$ is constant, $x$ is the dependent variable, $t$ is the independent variable\\
\textit{Eg:} $c\frac{\partial^2z}{\partial x^2} = \frac{\partial^2z}{\partial t^2}$, $c$ is constant, $z$ is the dependent variable, $t, x$ is the independent variable.

These are usually obvious:
\textit{Eg:} $\frac{dx}{dy} = 2$ or $\frac{dy}{dx} = \frac{1}{2}$ (which is usually dependent)

\textbf{Def \#3:} A DE is called an ordinary differential equation (ODE) to mean that it only contains ordinary derivatives.\\
A DE is called a partial differential equation (PDE) to mean that it contains partial derivatives.
\textit{Eg: ODE} $$\frac{d^2x}{dt^2} + sin(x)\frac{dx}{dt} + x^2 - sin(t^2) = 0$$\\
\textit{Eg: PDE} $$\frac{\partial z}{\partial x} + \frac{\partial^2 z}{\partial x \partial y} + \frac{\partial^3 z}{\partial t^3} + t^2 = 1$$
\textit{Note: ECE 205 has at least 10 weeks of ODE}\\
\newpage
\textbf{Def \#4:} The order of a DE is the highest derivative that appears. In general: higher order, means more difficult\\
\textit{Eg:} $$\frac{d^3x}{dt^3} + (\frac{d^2x}{dt^2})^2 + (\frac{dx}{dt})^{29} + sin(t) = 0$$

\textbf{Def \#5:} We say that a DE is linear to mean that all the terms in the dependent variable(s) are linear expressions\\
\textit{Eg:} Let $x = x(t)$, x is the dependent variable\\
Not linear
$$\frac{dx}{dt} + x^2 = 0$$
Linear
$$\frac{d^2x}{dt^2} + xt^2 = 0$$
Not linear (product of terms in x)
$$\frac{d^2x}{dt^2}\frac{dx}{dt} + sin(t) = 0$$
Linear
$$\frac{d^2x}{dt^2}+ sin(t)\frac{dx}{dt} + cos(t)x = t^2$$
Not linear
$$\frac{d^2x}{dt^2}+ sin(t)\frac{dx}{dt} + cos(tx) = t^2$$

In this course we will examine, 1st order ODE (4 types, one of them is linear), 2nd order linear
\newpage
\begin{center}
    {\LARGE 7th September 2018 \\ Lecture 2}
\end{center}'

\textbf{Def \#6:} The most general $n^{th}$ order ODE is of the form
$$a_n(t)\frac{d^nx}{dt^n} + a_{n-1}(t)\frac{d^{n-1}x}{dt^{n-1}} + a_{n-2}(t)\frac{d^{n-2}x}{dt^{n-2}} + ... + a_0(t)\frac{dx}{dt} + f(t) = 0$$

The coefficient function $a_n(t) \neq 0$, meaning it cannot be the $0^{th}$ function all the time, but it can have occasional zeros, \textit{Eg:} It can be $sin(t)$. If it is always zero then it will not be $n^{th}$ order anymore

\textbf{Def \#7:} We say that $n^th$ order ODE is homogeneous to mean that $f(t) \equiv 0$, \textit{Eg:} $a_1(t) \frac{dx}{dt} + a_0(t)x$

\textbf{Def \#8:} If the ODE is not homogeneous then we call it in-homogeneous or non-homogeneous

\textbf{Def \#9:} Given a linear in-homogeneous ODE, $$a_n(t)\frac{d^nx}{dt^n} + a_{n-1}(t)\frac{d^{n-1}x}{dt^{n-1}} + a_{n-2}(t)\frac{d^{n-2}x}{dt^{n-2}} + ... + a_0(t)\frac{dx}{dt} + f(t) = 0, \; f(t) \not\equiv 0$$
Then we call the ODE, $$a_n(t)\frac{d^nx}{dt^n} + a_{n-1}(t)\frac{d^{n-1}x}{dt^{n-1}} + a_{n-2}(t)\frac{d^{n-2}x}{dt^{n-2}} + ... + a_0(t)\frac{dx}{dt} + 0 = 0,$$ the associated homogeneous equation

\textbf{Def \#10:}\\
- A \textit{solution} of an ODE is a function that satisfies the ODE \\
- A \textit{solution set} of a DE is the set of all solutions of the DE \\
- The \textit{general solution} of a DE is the form of typical solutions


% The degree of a DE is the
\begin{center}
    {\LARGE 10th September 2018}
\end{center}'
$$\int_a^bf(x)dx = N\;\; a,\;b\; \epsilon\; \R \implies \frac{d}{dt}(\int_a^bf(x)dx) = \frac{dN}{dt} = 0$$
\\
$$\int^\tau sin(x)dx = g(\tau) \implies g(\tau) = -cos(\tau) + c \implies \frac{d}{d\tau} \int^\tau sin(x)dx = sin(\tau)$$

\end{document}
