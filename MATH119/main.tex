\documentclass[12pt]{article}

\usepackage{answers}
\usepackage{setspace}
\usepackage{graphicx}
\usepackage{enumitem}
\usepackage{multicol}
\usepackage{circuitikz}
\usepackage{adjustbox}
\usepackage{mathrsfs}
\usepackage{mathtools}
\usepackage{svg}
\usepackage[margin=0.5in]{geometry} 
\usepackage{amsmath,amsthm,amssymb}

\newcommand{\N}{\mathbb{N}}
\newcommand{\Z}{\mathbb{Z}}
\newcommand{\C}{\mathbb{C}}
\newcommand{\R}{\mathbb{R}}
\newcommand*{\Co}[2]{{}^{#1}C_{#2}}%

\DeclareMathOperator{\sech}{sech}
\DeclareMathOperator{\csch}{csch}
 
\newenvironment{theorem}[2][Theorem]{\begin{trivlist}
\item[\hskip \labelsep {\bfseries #1}\hskip \labelsep {\bfseries #2.}]}{\end{trivlist}}
\newenvironment{definition}[2][Definition]{\begin{trivlist}
\item[\hskip \labelsep {\bfseries #1}\hskip \labelsep {\bfseries #2.}]}{\end{trivlist}}
\newenvironment{proposition}[2][Proposition]{\begin{trivlist}
\item[\hskip \labelsep {\bfseries #1}\hskip \labelsep {\bfseries #2.}]}{\end{trivlist}}
\newenvironment{lemma}[2][Lemma]{\begin{trivlist}
\item[\hskip \labelsep {\bfseries #1}\hskip \labelsep {\bfseries #2.}]}{\end{trivlist}}
\newenvironment{exercise}[2][Exercise]{\begin{trivlist}
\item[\hskip \labelsep {\bfseries #1}\hskip \labelsep {\bfseries #2.}]}{\end{trivlist}}
\newenvironment{solution}[2][Solution]{\begin{trivlist}
\item[\hskip \labelsep {\bfseries #1}]}{\end{trivlist}}
\newenvironment{problem}[2][Problem]{\begin{trivlist}
\item[\hskip \labelsep {\bfseries #1}\hskip \labelsep {\bfseries #2.}]}{\end{trivlist}}
\newenvironment{question}[2][Question]{\begin{trivlist}
\item[\hskip \labelsep {\bfseries #1}\hskip \labelsep {\bfseries #2.}]}{\end{trivlist}}
\newenvironment{corollary}[2][Corollary]{\begin{trivlist}
\item[\hskip \labelsep {\bfseries #1}\hskip \labelsep {\bfseries #2.}]}{\end{trivlist}}
\pagenumbering{gobble}
\begin{document}
 
% --------------------------------------------------------------
%                         Start here
% -------------------------------------------------------------
\title{Points to Remember}%replace with the appropriate homework number
\author{Aditya Arora} %if necessary, replace with your course title
\maketitle
%Below is an example of the problem environment


\begin{enumerate}
\item Linearisation: $L(x) = y = f(a) + f'(a)(x-a)$
\item Bisection Method: Intermediate value theorem to approximate the root
\item Newton Raphson Procedure: $x_{n+1} = x_{n} - \frac{f(x_{n})}{f'(x_{n})}$
\item Taylor Theorem with Integral Remainder: $$f(x) = \sum_{k=0}^n \frac{f^{(k)}(x_0)}{k!}(x-x_0)^k + R_n(x)$$ 
where $$ R_n(x) = \int_{x_0}^x\frac{(x-t)^n}{n!}f^{(n+1)}(t) dt$$
\item Taylor's Inequality: The error in using an $n^{th}$-order polynomial $P_{n,x_0}(x)$ as an approximation to f(x) satisfies the inequality 
$$|R_n(x)| \leq K\frac{|x-x_0|^{n+1}}{(n+1)!}$$ where $|f^{n+1}(z)| \leq K$ for all values of $z$ between $x$ and $x_0$
\item Maclaurin Series
    \begin{enumerate}
        \parskip=0in
        \parsep=0in
        \itemsep=0.15in
    \item $\frac{1}{1-x} = \sum_{n=0}^{\infty}x^{n}$, for $|x| \le 1$
    \item $e^x = \sum_{n=0}^{\infty}\frac{x^{n}}{n!},$ for all $x$
    \item sin$x = \sum_{n=0}^{\infty}\frac{(-1)^n x^{2n+1}}{(2n+1)!},$ for all $x$
    \item cos$x = \sum_{n=0}^{\infty}\frac{(-1)^n x^{2n}}{(2n)!},$ for all $x$
    \item $(1+x)^k = \sum_{n=0}^{\infty}\Co{k}{n}x^{n},$ for all $x$
    \end{enumerate}
\item The Alternating Series Estimation Theorem:
Suppose the series $\sum_{n=0}^\infty (-1)^na_n$ can be shown to converge using the Alternating Series Test. Let S denote its sum. If $n=0$
we use the partial sum $S_n = a_0 + a_1 + . . . + a_n$ to approximate the value of $S$, then the error satisfies the inequality $|S-Sn| \leq a_{n+1}$. That is, the truncation error for an alternating series is no greater than the first term omitted.\newpage 
\item Integrals
    \begin{enumerate}
        \parskip=0in
        \parsep=0in
        \itemsep=0.15in
    \item $\int$ sec $x$ tan $x\ dx = $ sec $x \ + \ C  $
    \item $\int$ sec $x dx = $ ln$|$sec $x \ + $ tan $x| \ + \ C  $
    \end{enumerate}
\item Approximation of Integrals using Taylor Polynomials: $$\int_0^xf(t)dt = \int_0^xP_{n,l}(t)dt + \int_0^xR_n(t)dt$$ where $l$ is the center of the approximation defined appropriately. Then we can approximate $$\Bigl| \int_0^xR_n(t)dt\Bigl|\ \leq \int_0^x|R_n(t)|dt$$ The last line is only valid if $x > 0$ othewise one has to interchange the limits on the right hand side
\item Big-O: Given two functions $f$ and $g$ we say that "$f$ is an order of $g$ as $x\rightarrow x_0$" and write $f(x) =  \mathcal{O}(g(x))$ as $x\rightarrow x_0$ if there exists a constant $A$ greater than zero such that $|f(x)| \le A|g(x)|$
\item Multivariate Taylor Series: $$f(x,y) = f(P_0)$$ $$ + f_x(P_0)h + f_y(P_0)k $$ $$+ \frac{1}{2!}[f_{xx}(P_0)h^2 + 2f_xy(P_0)hk + f_{yy}(P_0)k^2]$$ $$ + \frac{1}{3!}[f_{xxx}(P_0)h^3 + 3f_{xxy}(P_0)h^2k + 3f_{xyy}(P_0)hk^2 + f_{yyy}(P_0)k^3]$$
\item Multivariate Linear Approximation: $\Delta f \approx f_x(P_0)\Delta x + f_y(P_0) \Delta y$
\item Gradient vector: $\nabla f = (\frac{\partial f}{\partial x}, \frac{\partial f}{\partial y})$. This changes chain rule to: $\nabla f(\vec r (t)) \cdot \vec r'(t)$ 
\item Directional derivatives: $D_{\vec u} f(a,b) = \nabla f(a,b) \cdot \vec u$
\item Un-constrained optimization: To locate the critical point set $\nabla f = \vec 0$. \\Then to classify the critical points calculate $D(x,y) = f_{xx}f_{yy} - (f_{xy})^2$
    \begin{enumerate}
        \item If $D(P_0) > 0$ then $f$ has an extremum at $P_0$
        \begin{enumerate}
            \item if $f_{xx}(P_0) < 0$ then maxima
            \item if $f_{xx}(P_0) > 0$ then minima
        \end{enumerate}
        \item If $D(P_0) < 0$ then $f$ does not have an extremum at $P_0$ (it is a saddle point instead)
        \item If $D(P_0) = 0$ then this test gives no conclusion
    \end{enumerate}
\item Method of Lagrange: To find the critical points of $f(x,y)$ subject to constraint $g(x,y) = K$ where $K$ is a constant, find the values of $x$ and $y$ for which $\nabla f = \lambda \nabla g$ and $g(x,y) = K$, for some constant K
\end{enumerate}



\pagebreak
\setlength{\voffset}{0.15in}

\end{document}
