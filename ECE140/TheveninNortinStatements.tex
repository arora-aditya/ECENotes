\documentclass[12pt]{article}

\usepackage{answers}
\usepackage{setspace}
\usepackage{graphicx}
\usepackage{enumitem}
\usepackage{multicol}
\usepackage{mathrsfs}
\usepackage[margin=1in]{geometry} 
\usepackage{amsmath,amsthm,amssymb}
\setlength{\hoffset}{0pt}
\setlength{\headsep}{5pt}
\newcommand{\N}{\mathbb{N}}
\newcommand{\Z}{\mathbb{Z}}
\newcommand{\C}{\mathbb{C}}
\newcommand{\R}{\mathbb{R}}

\DeclareMathOperator{\sech}{sech}
\DeclareMathOperator{\csch}{csch}
 
\newenvironment{theorem}[2][Theorem]{\begin{trivlist}
\item[\hskip \labelsep {\bfseries #1}\hskip \labelsep {\bfseries #2.}]}{\end{trivlist}}
\newenvironment{definition}[2][Definition]{\begin{trivlist}
\item[\hskip \labelsep {\bfseries #1}\hskip \labelsep {\bfseries #2.}]}{\end{trivlist}}
\newenvironment{proposition}[2][Proposition]{\begin{trivlist}
\item[\hskip \labelsep {\bfseries #1}\hskip \labelsep {\bfseries #2.}]}{\end{trivlist}}
\newenvironment{lemma}[2][Lemma]{\begin{trivlist}
\item[\hskip \labelsep {\bfseries #1}\hskip \labelsep {\bfseries #2.}]}{\end{trivlist}}
\newenvironment{exercise}[2][Exercise]{\begin{trivlist}
\item[\hskip \labelsep {\bfseries #1}\hskip \labelsep {\bfseries #2.}]}{\end{trivlist}}
\newenvironment{solution}[2][Solution]{\begin{trivlist}
\item[\hskip \labelsep {\bfseries #1}]}{\end{trivlist}}
\newenvironment{problem}[2][Problem]{\begin{trivlist}
\item[\hskip \labelsep {\bfseries #1}\hskip \labelsep {\bfseries #2.}]}{\end{trivlist}}
\newenvironment{question}[2][Question]{\begin{trivlist}
\item[\hskip \labelsep {\bfseries #1}\hskip \labelsep {\bfseries #2.}]}{\end{trivlist}}
\newenvironment{corollary}[2][Corollary]{\begin{trivlist}
\item[\hskip \labelsep {\bfseries #1}\hskip \labelsep {\bfseries #2.}]}{\end{trivlist}}
\pagenumbering{gobble}
\begin{document}
 
% --------------------------------------------------------------
%                         Start here
% --------------------------------------------------------------
 
\title{Thevenin and Norton's Theorems}%replace with the appropriate homework number
\author{Aditya Arora\\ %replace with your name
Winter 2018} %if necessary, replace with your course title
 
\maketitle
%Below is an example of the problem environment

\begin{center}
   STATEMENTS
\end{center}
\vskip 0.1in
\textbf{A Statement of Thévenin’s Theorem}
\begin{trivlist}
\item1. Given any linear circuit, rearrange it in the form of two networks, A and B, connected by two wires. Network A is the network to be simplified; B will be left untouched.
\item2. Disconnect network B. Define a voltage $v_{oc}$ as the voltage now appearing across the terminals of network A.
\item3. Turn off or “zero out” every independent source in network A to form an inactive network. Leave dependent sources unchanged.
\item4. Connect an independent voltage source with value $v_{oc}$ in series with the inactive network. Do not complete the circuit; leave the two terminals disconnected.
\item5. Connect network B to the terminals of the new network A. All currents and voltages in B will remain unchanged.
\end{trivlist}
\vskip 0.1in
\begin{trivlist}
    \item \textbf{CASE 1:} If the network has no dependent sources, we turn off all the independent sources. 
    \item \textbf{CASE 2:} If the network has dependent sources we turn off all independent sources. We can either then choose to apply a 1V source or a 1A source across the terminals and find $R_{th}$ accordingly using the help of other circuit theorems
\end{trivlist}

\newpage
\textbf{A Statement of Norton’s Theorem}
\begin{trivlist}
\item1. Given any linear circuit, rearrange it in the form of two networks, A and B, connected by two wires. Network A is the network to be simplified; B will be left untouched. As before, if either network contains a dependent source, its controlling variable must be in the same network.
\item2. Disconnect network B, and short the terminals of A. Define a current $i_{sc}$ as the current now flowing through the shorted terminals of network A.
\item3. Turn off or “zero out” every independent source in network A to form an inactive network. Leave dependent sources unchanged.
\item4. Connect an independent current source with value $i_{sc}$ in parallel with the inactive network. Do not complete the circuit; leave the two terminals disconnected.
\item5. Connect network B to the terminals of the new network A. All currents and voltages in B will remain unchanged.
\end{trivlist}
\vskip 0.1in
We can find any two of the following three to determine the Norton equivalent of a circuit:
\begin{enumerate}
    \item The open-circuit voltage $v_{oc}$ across terminals $a$ and $b$ (no current flows out from this open circuit)
    \item The short-circuit current $i_{sc}$ at terminals $a$ and $b$
    \item The equivalent or input resistance at terminals $a$ and $b$ when all the independent sources are turned off
\end{enumerate}
Using these we can find the Norton's equivalent since: 
$$I_{N} = i_{sc}, V_{Th} = v_{oc}, R_{Th} = \frac{v_{oc}}{i_{sc}} = R_{N}$$

\end{document}
